\documentclass[12pt]{article}

% ------------------------------------------------------------
% Arabic & Font Setup (Fixed)
% ------------------------------------------------------------
\usepackage{fontspec}
\usepackage{polyglossia}
\setdefaultlanguage{english}
\setotherlanguage{arabic}

% Arabic font
\newfontfamily\arabicfont[
	Script=Arabic,
	Scale=1.1
]{Noto Naskh Arabic}

% ------------------------------------------------------------
% General Layout
% ------------------------------------------------------------
\usepackage{geometry}
\geometry{margin=1in}

\usepackage{setspace}
\setstretch{1.15}

\usepackage{titlesec}
\titleformat{\section}{\normalfont\Large\bfseries}{}{0em}{}
\titleformat{\subsection}{\normalfont\large\bfseries}{}{0em}{}

\usepackage{marginnote}
\renewcommand*{\marginfont}{\footnotesize\itshape}

\usepackage{csquotes}
\usepackage{hyperref}
\hypersetup{colorlinks=true,linkcolor=black,urlcolor=black}

% ------------------------------------------------------------
% Title Page
% ------------------------------------------------------------
\title{\Huge \textbf{PROTOCOL ZERO}\\[6pt]
\large The Algorithmic Deconstruction of Arabic\\[2pt]
\large A Manifesto in Ten Lectures}

\author{Flyxion}
\date{}

% ------------------------------------------------------------
% Document Begins
% ------------------------------------------------------------
\begin{document}
\maketitle
\thispagestyle{empty}

\vspace{1in}

\begin{center}
\textit{“Language is not a human invention. It is a celestial machine.  
Every tongue is a failed attempt to describe the architecture of the world.”}\\
--- Giordano Bruno
\end{center}

\newpage
\tableofcontents
\newpage


%%%%%%%%%%%%%%%%%%%%%%%%%%%%%%%%%%%%%%%%%%%%%%%%%%%%%%%%%%%%%%
% LECTURE 1 — THE SKELETON KEY
%%%%%%%%%%%%%%%%%%%%%%%%%%%%%%%%%%%%%%%%%%%%%%%%%%%%%%%%%%%%%%

\section*{Lecture 01: The Skeleton Key}
\addcontentsline{toc}{section}{Lecture 01: The Skeleton Key}

\marginpar{The Null Principle:\\ Meaning precedes sound.}

\noindent
The first task of this manifesto is to dismantle the falsehood that Arabic is a language like any other, a mere collection of words and rules gathered into a dictionary by historical accident. Such a belief is convenient for departments of linguistics, reassuring for pedagogues, and comforting for students who prefer a world where language is arbitrary and nothing is required of them but memorization. Yet this belief is wrong. Arabic is not arbitrary, nor accidental, nor even truly historical. It is engineered. It is a structure far older than its surviving texts, a computational architecture masquerading as a human tongue.

To enter Arabic, one must first abandon the idea that words are fundamental units. Words are surface phenomena, temporary crystallizations of a deeper system. The true atoms of meaning are the roots: tri-consonantal skeletons that encode conceptual spaces rather than lexical items.

\subsection*{Roots as Hardware}

Consider the root \textarabic{ك ت ب}. In its raw state, it does not mean “write,” “book,” “scribe,” or “library,” but a deeper conceptual field: inscription, imprint, the transfer of mind into matter.  
Every derivative word is an excitation of this skeleton.

Arabic is morphology as chemistry.

\subsection*{Patterns as Chemical Reactions}

The patterns, or \textarabic{الأوزان}, are algorithms. Passing a root through a pattern yields predictable, lawful transformations: causation, reciprocity, reflexivity, intensity, seeking, becoming.

\subsection*{Vowels as Energy States}

Vowels do not decorate; they direct.  
A fatḥa propels.  
A kasra contracts.  
A ḍamma encloses and strengthens.

To change a vowel is to change the state of being.

\newpage


%%%%%%%%%%%%%%%%%%%%%%%%%%%%%%%%%%%%%%%%%%%%%%%%%%%%%%%%%%%%%%
% LECTURE 2 — VOWEL VECTORS & THE SYNTACTIC CIRCUIT
%%%%%%%%%%%%%%%%%%%%%%%%%%%%%%%%%%%%%%%%%%%%%%%%%%%%%%%%%%%%%%

\section*{Lecture 02: Vowel Vectors and the Syntactic Circuit}
\addcontentsline{toc}{section}{Lecture 02}

\marginpar{A vowel is a vector.}

\noindent
If consonants form the structure, vowels form the current. They energize the language. They are not static marks but forces.

\subsection*{The Three Motions}

The shift:

\begin{center}
\textarabic{كَتَبَ} → \textarabic{كُتِبَ}
\end{center}

is not mechanical.  
It is a reorientation of energy.

\subsection*{Case Endings as Gravitational States}

Nominative is sovereignty, marked by \textarabic{ـُ}.  
Accusative is exposure, marked by \textarabic{ـَ}.  
Genitive is relational gravity, marked by \textarabic{ـِ}.

\subsection*{The Syntactic Circuit}

Arabic sentences behave like electric circuits: nodes of activation, vectors of flow, points of resistance, lines of force.

\newpage


%%%%%%%%%%%%%%%%%%%%%%%%%%%%%%%%%%%%%%%%%%%%%%%%%%%%%%%%%%%%%%
% LECTURE 3 — PATTERNS AS ALGORITHMS
%%%%%%%%%%%%%%%%%%%%%%%%%%%%%%%%%%%%%%%%%%%%%%%%%%%%%%%%%%%%%%

\section*{Lecture 03: Patterns as Algorithms}
\addcontentsline{toc}{section}{Lecture 03}

\marginpar{Patterns are executable code.}

\noindent
Patterns are morphological functions into which roots are passed. They do not “contain meaning”; they perform meaning.

\subsection*{Examples}

\textarabic{فَعَّلَ} intensifies or causes.  
\textarabic{تَفَاعَلَ} creates reciprocity.  
\textarabic{اِسْتَفْعَلَ} encodes seeking or requesting.

Patterns are not templates. They are transformations.

\newpage


%%%%%%%%%%%%%%%%%%%%%%%%%%%%%%%%%%%%%%%%%%%%%%%%%%%%%%%%%%%%%%
% LECTURE 4 — THE SYNTAX OF POWER
%%%%%%%%%%%%%%%%%%%%%%%%%%%%%%%%%%%%%%%%%%%%%%%%%%%%%%%%%%%%%%

\section*{Lecture 04: The Syntax of Power}
\addcontentsline{toc}{section}{Lecture 04}

\marginpar{Syntax is politics.}

\noindent
Arabic makes visible the invisible politics embedded in syntax.  
Every noun has status.  
Every particle has authority.  
Every position in the sentence is a throne or a chain.

\subsection*{Operators}

Particles like  
\textarabic{إِنَّ}  
\textarabic{كان}  
\textarabic{لن}  
\textarabic{لم}  

do not simply modify grammar; they modify reality.

\newpage


%%%%%%%%%%%%%%%%%%%%%%%%%%%%%%%%%%%%%%%%%%%%%%%%%%%%%%%%%%%%%%
% LECTURE 5 — THE BEINGLESS SENTENCE
%%%%%%%%%%%%%%%%%%%%%%%%%%%%%%%%%%%%%%%%%%%%%%%%%%%%%%%%%%%%%%

\section*{Lecture 05: The Beingless Sentence}
\addcontentsline{toc}{section}{Lecture 05}

\marginpar{Existence is assumed.}

\noindent
Arabic has no present-tense “to be.”  
This is not omission. It is ontology.

\begin{center}
\textarabic{الكتابُ جديدٌ}
\end{center}

does not mean “the book \textit{is} new.”  
It means: “The book — new.”  
Essence meets attribute with nothing between.

\newpage


%%%%%%%%%%%%%%%%%%%%%%%%%%%%%%%%%%%%%%%%%%%%%%%%%%%%%%%%%%%%%%
% LECTURE 6 — THE HIDDEN GEOMETRY OF SENTENCES
%%%%%%%%%%%%%%%%%%%%%%%%%%%%%%%%%%%%%%%%%%%%%%%%%%%%%%%%%%%%%%

\section*{Lecture 06: The Hidden Geometry of Sentences}
\addcontentsline{toc}{section}{Lecture 06}

\marginpar{Sentences are shapes.}

\noindent
Arabic sentences are not lines. They are figures: vectors, axes, orbits, distortions, ellipses of meaning.

\newpage


%%%%%%%%%%%%%%%%%%%%%%%%%%%%%%%%%%%%%%%%%%%%%%%%%%%%%%%%%%%%%%
% LECTURE 7 — THE MEDITERRANEAN COMPILER FARM
%%%%%%%%%%%%%%%%%%%%%%%%%%%%%%%%%%%%%%%%%%%%%%%%%%%%%%%%%%%%%%

\section*{Lecture 07: The Mediterranean Compiler Farm}
\addcontentsline{toc}{section}{Lecture 07}

\marginpar{The sea compiled languages.}

\noindent
Arabic is the final stable release of a 3,000-year linguistic compression process.

\newpage


%%%%%%%%%%%%%%%%%%%%%%%%%%%%%%%%%%%%%%%%%%%%%%%%%%%%%%%%%%%%%%
% LECTURE 8 — NUMBERS & SUKŪN AS ZERO
%%%%%%%%%%%%%%%%%%%%%%%%%%%%%%%%%%%%%%%%%%%%%%%%%%%%%%%%%%%%%%

\section*{Lecture 08: Numbers, Tally Marks, and Sukūn as Zero}
\addcontentsline{toc}{section}{Lecture 08}

\marginpar{Zero is silence.}

\noindent
Before numerical zero, Arabic already encoded nullity:  
the \textarabic{سُكُون}.

Sukūn is the first zero:  
the extinguished vowel,  
the collapse of motion,  
the return to potential.

\newpage


%%%%%%%%%%%%%%%%%%%%%%%%%%%%%%%%%%%%%%%%%%%%%%%%%%%%%%%%%%%%%%
% LECTURE 9 — SUN & MOON LETTERS AND THE MOUTH MAP
%%%%%%%%%%%%%%%%%%%%%%%%%%%%%%%%%%%%%%%%%%%%%%%%%%%%%%%%%%%%%%

\section*{Lecture 09: Sun Letters, Moon Letters, and the Mouth Map}
\addcontentsline{toc}{section}{Lecture 09}

\marginpar{The mouth is a cosmos.}

\noindent
Sun letters burn through the \textarabic{ل} of \textarabic{ال}.  
Moon letters reveal it.  
This is not arbitrary—it is a map of articulation and a cosmology.

\subsection*{Example}

\textarabic{الشَّمْسُ}  
(assimilation)

\textarabic{القَمَرُ}  
(non-assimilation)

\newpage


%%%%%%%%%%%%%%%%%%%%%%%%%%%%%%%%%%%%%%%%%%%%%%%%%%%%%%%%%%%%%%
% LECTURE 10 — GIORDANO BRUNO'S ARK
%%%%%%%%%%%%%%%%%%%%%%%%%%%%%%%%%%%%%%%%%%%%%%%%%%%%%%%%%%%%%%

\section*{Lecture 10: Giordano Bruno’s Ark}
\addcontentsline{toc}{section}{Lecture 10}

\marginpar{Categories are politics.}

\noindent
Bruno’s Ark is not zoology.  
It is classification — an early structuralism.  
Arabic’s root system mirrors Bruno’s combinatorial ontology.

\newpage


%%%%%%%%%%%%%%%%%%%%%%%%%%%%%%%%%%%%%%%%%%%%%%%%%%%%%%%%%%%%%%
% EPILOGUE
%%%%%%%%%%%%%%%%%%%%%%%%%%%%%%%%%%%%%%%%%%%%%%%%%%%%%%%%%%%%%%

\section*{Epilogue: The Null Hypothesis of Language}
\addcontentsline{toc}{section}{Epilogue}

\marginpar{Nothing is accidental.}

\noindent
Arabic is not a language.  
It is an engine of meaning.

Roots = conceptual spaces  
Patterns = transformations  
Vowels = energy  
Syntax = power  
Phonology = cosmology

Protocol Zero ends here.  
Your work begins here.

\end{document}

